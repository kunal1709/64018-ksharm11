% Options for packages loaded elsewhere
\PassOptionsToPackage{unicode}{hyperref}
\PassOptionsToPackage{hyphens}{url}
%
\documentclass[
]{article}
\usepackage{amsmath,amssymb}
\usepackage{lmodern}
\usepackage{ifxetex,ifluatex}
\ifnum 0\ifxetex 1\fi\ifluatex 1\fi=0 % if pdftex
  \usepackage[T1]{fontenc}
  \usepackage[utf8]{inputenc}
  \usepackage{textcomp} % provide euro and other symbols
\else % if luatex or xetex
  \usepackage{unicode-math}
  \defaultfontfeatures{Scale=MatchLowercase}
  \defaultfontfeatures[\rmfamily]{Ligatures=TeX,Scale=1}
\fi
% Use upquote if available, for straight quotes in verbatim environments
\IfFileExists{upquote.sty}{\usepackage{upquote}}{}
\IfFileExists{microtype.sty}{% use microtype if available
  \usepackage[]{microtype}
  \UseMicrotypeSet[protrusion]{basicmath} % disable protrusion for tt fonts
}{}
\makeatletter
\@ifundefined{KOMAClassName}{% if non-KOMA class
  \IfFileExists{parskip.sty}{%
    \usepackage{parskip}
  }{% else
    \setlength{\parindent}{0pt}
    \setlength{\parskip}{6pt plus 2pt minus 1pt}}
}{% if KOMA class
  \KOMAoptions{parskip=half}}
\makeatother
\usepackage{xcolor}
\IfFileExists{xurl.sty}{\usepackage{xurl}}{} % add URL line breaks if available
\IfFileExists{bookmark.sty}{\usepackage{bookmark}}{\usepackage{hyperref}}
\hypersetup{
  pdftitle={Assignment 3 Weigelt Corporation},
  hidelinks,
  pdfcreator={LaTeX via pandoc}}
\urlstyle{same} % disable monospaced font for URLs
\usepackage[margin=1in]{geometry}
\usepackage{color}
\usepackage{fancyvrb}
\newcommand{\VerbBar}{|}
\newcommand{\VERB}{\Verb[commandchars=\\\{\}]}
\DefineVerbatimEnvironment{Highlighting}{Verbatim}{commandchars=\\\{\}}
% Add ',fontsize=\small' for more characters per line
\usepackage{framed}
\definecolor{shadecolor}{RGB}{248,248,248}
\newenvironment{Shaded}{\begin{snugshade}}{\end{snugshade}}
\newcommand{\AlertTok}[1]{\textcolor[rgb]{0.94,0.16,0.16}{#1}}
\newcommand{\AnnotationTok}[1]{\textcolor[rgb]{0.56,0.35,0.01}{\textbf{\textit{#1}}}}
\newcommand{\AttributeTok}[1]{\textcolor[rgb]{0.77,0.63,0.00}{#1}}
\newcommand{\BaseNTok}[1]{\textcolor[rgb]{0.00,0.00,0.81}{#1}}
\newcommand{\BuiltInTok}[1]{#1}
\newcommand{\CharTok}[1]{\textcolor[rgb]{0.31,0.60,0.02}{#1}}
\newcommand{\CommentTok}[1]{\textcolor[rgb]{0.56,0.35,0.01}{\textit{#1}}}
\newcommand{\CommentVarTok}[1]{\textcolor[rgb]{0.56,0.35,0.01}{\textbf{\textit{#1}}}}
\newcommand{\ConstantTok}[1]{\textcolor[rgb]{0.00,0.00,0.00}{#1}}
\newcommand{\ControlFlowTok}[1]{\textcolor[rgb]{0.13,0.29,0.53}{\textbf{#1}}}
\newcommand{\DataTypeTok}[1]{\textcolor[rgb]{0.13,0.29,0.53}{#1}}
\newcommand{\DecValTok}[1]{\textcolor[rgb]{0.00,0.00,0.81}{#1}}
\newcommand{\DocumentationTok}[1]{\textcolor[rgb]{0.56,0.35,0.01}{\textbf{\textit{#1}}}}
\newcommand{\ErrorTok}[1]{\textcolor[rgb]{0.64,0.00,0.00}{\textbf{#1}}}
\newcommand{\ExtensionTok}[1]{#1}
\newcommand{\FloatTok}[1]{\textcolor[rgb]{0.00,0.00,0.81}{#1}}
\newcommand{\FunctionTok}[1]{\textcolor[rgb]{0.00,0.00,0.00}{#1}}
\newcommand{\ImportTok}[1]{#1}
\newcommand{\InformationTok}[1]{\textcolor[rgb]{0.56,0.35,0.01}{\textbf{\textit{#1}}}}
\newcommand{\KeywordTok}[1]{\textcolor[rgb]{0.13,0.29,0.53}{\textbf{#1}}}
\newcommand{\NormalTok}[1]{#1}
\newcommand{\OperatorTok}[1]{\textcolor[rgb]{0.81,0.36,0.00}{\textbf{#1}}}
\newcommand{\OtherTok}[1]{\textcolor[rgb]{0.56,0.35,0.01}{#1}}
\newcommand{\PreprocessorTok}[1]{\textcolor[rgb]{0.56,0.35,0.01}{\textit{#1}}}
\newcommand{\RegionMarkerTok}[1]{#1}
\newcommand{\SpecialCharTok}[1]{\textcolor[rgb]{0.00,0.00,0.00}{#1}}
\newcommand{\SpecialStringTok}[1]{\textcolor[rgb]{0.31,0.60,0.02}{#1}}
\newcommand{\StringTok}[1]{\textcolor[rgb]{0.31,0.60,0.02}{#1}}
\newcommand{\VariableTok}[1]{\textcolor[rgb]{0.00,0.00,0.00}{#1}}
\newcommand{\VerbatimStringTok}[1]{\textcolor[rgb]{0.31,0.60,0.02}{#1}}
\newcommand{\WarningTok}[1]{\textcolor[rgb]{0.56,0.35,0.01}{\textbf{\textit{#1}}}}
\usepackage{graphicx}
\makeatletter
\def\maxwidth{\ifdim\Gin@nat@width>\linewidth\linewidth\else\Gin@nat@width\fi}
\def\maxheight{\ifdim\Gin@nat@height>\textheight\textheight\else\Gin@nat@height\fi}
\makeatother
% Scale images if necessary, so that they will not overflow the page
% margins by default, and it is still possible to overwrite the defaults
% using explicit options in \includegraphics[width, height, ...]{}
\setkeys{Gin}{width=\maxwidth,height=\maxheight,keepaspectratio}
% Set default figure placement to htbp
\makeatletter
\def\fps@figure{htbp}
\makeatother
\setlength{\emergencystretch}{3em} % prevent overfull lines
\providecommand{\tightlist}{%
  \setlength{\itemsep}{0pt}\setlength{\parskip}{0pt}}
\setcounter{secnumdepth}{-\maxdimen} % remove section numbering
\ifluatex
  \usepackage{selnolig}  % disable illegal ligatures
\fi

\title{Assignment 3 Weigelt Corporation}
\author{}
\date{\vspace{-2.5em}}

\begin{document}
\maketitle

\hypertarget{question}{%
\subsection{Question:}\label{question}}

The WeigeltCorporation has three branch plants with excess production
capacity. Fortunately, the corporation has a new product ready to begin
production, and all three plants have this capability, so some of the
excess capacity can be used in this way. This product can be made in
three sizes--large, medium, and small--that yield a net unit profit of
\$420, \$360, and \$300, respectively. Plants 1, 2, and 3 have the
excess capacity to produce 750, 900, and 450 units per day of this
product, respectively, regardless of the size or combination of sizes
involved.

The amount of available in-process storage space also imposes a
limitation on the production rates of the new product. Plants 1, 2, and
3 have 13,000, 12,000, and 5,000 square feet, respectively, of
in-process storage space available for a day's production of this
product. Each unit of the large, medium, and small sizes produced per
day requires 20, 15, and 12 square feet, respectively.

Sales forecasts indicate that if available, 900, 1,200, and 750 units of
the large, medium, and small sizes, respectively, would be sold per day.

At each plant, some employees will need to be laid off unless most of
the plant's excess production capacity can be used to produce the new
product. To avoid layoffs if possible, management has decided that the
plants should use the same percentage of their excess capacity to
produce the new product.

Management wishes to know how much of each of the sizes should be
produced by each of the plants to maximize profit.

1.Solve the problem usinglpsolve, or any other equivalent library in R.

2.Identify the shadow prices, dual solution, and reduced costs

3.Further, identify the sensitivity of the above prices and costs. That
is, specify the range of shadow prices and reduced cost within which the
optimal solution will not change.

4.Formulate the dual of the above problem and solve it. Does the
solution agree with what you observed for the primal problem?

\begin{Shaded}
\begin{Highlighting}[]
\FunctionTok{library}\NormalTok{(}\StringTok{\textquotesingle{}lpSolveAPI\textquotesingle{}}\NormalTok{)}
\end{Highlighting}
\end{Shaded}

\begin{verbatim}
## Warning: package 'lpSolveAPI' was built under R version 4.1.1
\end{verbatim}

\hypertarget{to-read-the-lp-file.}{%
\subsection{To read the LP file.}\label{to-read-the-lp-file.}}

\begin{Shaded}
\begin{Highlighting}[]
\NormalTok{WC }\OtherTok{\textless{}{-}} \FunctionTok{read.lp}\NormalTok{(}\StringTok{"Weigelt.lp"}\NormalTok{)}

\NormalTok{WC}
\end{Highlighting}
\end{Shaded}

\begin{verbatim}
## Model name: 
##   a linear program with 9 decision variables and 11 constraints
\end{verbatim}

\#\#To solve the LP.

\begin{Shaded}
\begin{Highlighting}[]
\FunctionTok{solve}\NormalTok{(WC)}
\end{Highlighting}
\end{Shaded}

\begin{verbatim}
## [1] 0
\end{verbatim}

\#\#To compute the objective function value.

\begin{Shaded}
\begin{Highlighting}[]
\FunctionTok{get.objective}\NormalTok{(WC)}
\end{Highlighting}
\end{Shaded}

\begin{verbatim}
## [1] 696000
\end{verbatim}

\#\#To compute the values of decision variables.

\begin{Shaded}
\begin{Highlighting}[]
\FunctionTok{get.variables}\NormalTok{(WC)}
\end{Highlighting}
\end{Shaded}

\begin{verbatim}
## [1] 516.6667 177.7778   0.0000   0.0000 666.6667 166.6667   0.0000   0.0000
## [9] 416.6667
\end{verbatim}

\#\#To compute the values of constraints.

\begin{Shaded}
\begin{Highlighting}[]
\FunctionTok{get.constraints}\NormalTok{(WC)}
\end{Highlighting}
\end{Shaded}

\begin{verbatim}
##  [1]  6.944444e+02  8.333333e+02  4.166667e+02  1.300000e+04  1.200000e+04
##  [6]  5.000000e+03  5.166667e+02  8.444444e+02  5.833333e+02 -2.037268e-10
## [11]  0.000000e+00
\end{verbatim}

\#\#To compute Shadow price

\begin{Shaded}
\begin{Highlighting}[]
\FunctionTok{get.sensitivity.rhs}\NormalTok{(WC)}\SpecialCharTok{$}\NormalTok{duals[}\DecValTok{1}\SpecialCharTok{:}\DecValTok{11}\NormalTok{]}
\end{Highlighting}
\end{Shaded}

\begin{verbatim}
##  [1]  0.00  0.00  0.00 12.00 20.00 60.00  0.00  0.00  0.00 -0.08  0.56
\end{verbatim}

\#\#To compute Reduced cost.

\begin{Shaded}
\begin{Highlighting}[]
\FunctionTok{get.sensitivity.rhs}\NormalTok{(WC)}\SpecialCharTok{$}\NormalTok{duals[}\DecValTok{12}\SpecialCharTok{:}\DecValTok{20}\NormalTok{]}
\end{Highlighting}
\end{Shaded}

\begin{verbatim}
## [1]    0    0  -24  -40    0    0 -360 -120    0
\end{verbatim}

\#\#To compute the dual solution

\begin{Shaded}
\begin{Highlighting}[]
\FunctionTok{get.dual.solution}\NormalTok{(WC)}
\end{Highlighting}
\end{Shaded}

\begin{verbatim}
##  [1]    1.00    0.00    0.00    0.00   12.00   20.00   60.00    0.00    0.00
## [10]    0.00   -0.08    0.56    0.00    0.00  -24.00  -40.00    0.00    0.00
## [19] -360.00 -120.00    0.00
\end{verbatim}

\begin{Shaded}
\begin{Highlighting}[]
\FunctionTok{get.sensitivity.rhs}\NormalTok{(WC)}
\end{Highlighting}
\end{Shaded}

\begin{verbatim}
## $duals
##  [1]    0.00    0.00    0.00   12.00   20.00   60.00    0.00    0.00    0.00
## [10]   -0.08    0.56    0.00    0.00  -24.00  -40.00    0.00    0.00 -360.00
## [19] -120.00    0.00
## 
## $dualsfrom
##  [1] -1.000000e+30 -1.000000e+30 -1.000000e+30  1.122222e+04  1.150000e+04
##  [6]  4.800000e+03 -1.000000e+30 -1.000000e+30 -1.000000e+30 -2.500000e+04
## [11] -1.250000e+04 -1.000000e+30 -1.000000e+30 -2.222222e+02 -1.000000e+02
## [16] -1.000000e+30 -1.000000e+30 -2.000000e+01 -4.444444e+01 -1.000000e+30
## 
## $dualstill
##  [1] 1.000000e+30 1.000000e+30 1.000000e+30 1.388889e+04 1.250000e+04
##  [6] 5.181818e+03 1.000000e+30 1.000000e+30 1.000000e+30 2.500000e+04
## [11] 1.250000e+04 1.000000e+30 1.000000e+30 1.111111e+02 1.000000e+02
## [16] 1.000000e+30 1.000000e+30 2.500000e+01 6.666667e+01 1.000000e+30
\end{verbatim}

\#\#To compute the range of shadow price for which the optimal solution
remains unchanged.

\begin{Shaded}
\begin{Highlighting}[]
\FunctionTok{cbind}\NormalTok{(}\AttributeTok{Shadow\_Price=}\FunctionTok{get.sensitivity.rhs}\NormalTok{(WC)}\SpecialCharTok{$}\NormalTok{duals[}\DecValTok{1}\SpecialCharTok{:}\DecValTok{11}\NormalTok{], }\AttributeTok{Lower\_bound=}\FunctionTok{get.sensitivity.rhs}\NormalTok{(WC)}\SpecialCharTok{$}\NormalTok{dualsfrom[}\DecValTok{1}\SpecialCharTok{:}\DecValTok{11}\NormalTok{], }\AttributeTok{Upper\_bound=}\FunctionTok{get.sensitivity.rhs}\NormalTok{(WC)}\SpecialCharTok{$}\NormalTok{dualstill[}\DecValTok{1}\SpecialCharTok{:}\DecValTok{11}\NormalTok{])}
\end{Highlighting}
\end{Shaded}

\begin{verbatim}
##       Shadow_Price   Lower_bound  Upper_bound
##  [1,]         0.00 -1.000000e+30 1.000000e+30
##  [2,]         0.00 -1.000000e+30 1.000000e+30
##  [3,]         0.00 -1.000000e+30 1.000000e+30
##  [4,]        12.00  1.122222e+04 1.388889e+04
##  [5,]        20.00  1.150000e+04 1.250000e+04
##  [6,]        60.00  4.800000e+03 5.181818e+03
##  [7,]         0.00 -1.000000e+30 1.000000e+30
##  [8,]         0.00 -1.000000e+30 1.000000e+30
##  [9,]         0.00 -1.000000e+30 1.000000e+30
## [10,]        -0.08 -2.500000e+04 2.500000e+04
## [11,]         0.56 -1.250000e+04 1.250000e+04
\end{verbatim}

\#\#To compute the range of shadow price for which the optimal solution
remains unchanged.

\begin{Shaded}
\begin{Highlighting}[]
\FunctionTok{cbind}\NormalTok{(}\AttributeTok{Reduced\_Cost=}\FunctionTok{get.sensitivity.rhs}\NormalTok{(WC)}\SpecialCharTok{$}\NormalTok{duals[}\DecValTok{12}\SpecialCharTok{:}\DecValTok{20}\NormalTok{], }\AttributeTok{Lower\_bound=}\FunctionTok{get.sensitivity.rhs}\NormalTok{(WC)}\SpecialCharTok{$}\NormalTok{dualsfrom[}\DecValTok{12}\SpecialCharTok{:}\DecValTok{20}\NormalTok{], }\AttributeTok{Upper\_bound=}\FunctionTok{get.sensitivity.rhs}\NormalTok{(WC)}\SpecialCharTok{$}\NormalTok{dualstill[}\DecValTok{12}\SpecialCharTok{:}\DecValTok{20}\NormalTok{])}
\end{Highlighting}
\end{Shaded}

\begin{verbatim}
##       Reduced_Cost   Lower_bound  Upper_bound
##  [1,]            0 -1.000000e+30 1.000000e+30
##  [2,]            0 -1.000000e+30 1.000000e+30
##  [3,]          -24 -2.222222e+02 1.111111e+02
##  [4,]          -40 -1.000000e+02 1.000000e+02
##  [5,]            0 -1.000000e+30 1.000000e+30
##  [6,]            0 -1.000000e+30 1.000000e+30
##  [7,]         -360 -2.000000e+01 2.500000e+01
##  [8,]         -120 -4.444444e+01 6.666667e+01
##  [9,]            0 -1.000000e+30 1.000000e+30
\end{verbatim}

\#\#The Dual of the Weiglt Coporation problem: Objective function:

minz = 750 y1 + 900 y2 + 450 y3 + 13000 y4 + 12000 y5 + 5000 y6 + 900 y7
+ 1200 y8 + 750 y9;

Subject to

y1 + 20 y4 + y7 + 900 y10 + 450 y11 \textgreater= 420;

y1 + 15 y4 + y8 + 900 y10 + 450 y11 \textgreater= 360;

y1 + 12 y4 + y9 + 900 y10 + 450 y11 \textgreater= 300;

y2 + 20 y5 + y7 - 750 y10 \textgreater= 420;

y2 + 15 y5 + y8 - 750 y10 \textgreater= 360;

y2 + 12 y5 + y9 - 750 y10 \textgreater= 300;

y3 + 20 y6 + y7 - 750 y11 \textgreater= 420;

y3 + 15 y6 + y8 - 750 y11 \textgreater= 360;

y3 + 12 y6 + y9 - 750 y11 \textgreater= 300;

And

y1, y2, \ldots, y9 \textgreater= 0;

y10, y11 unrestricted;

Thus, Lets replace y10 = y10\_1 - y10\_2 and y11 = y11\_1 - y11\_2,
where y10\_1, y10\_2, y11\_1, y11\_2 \textgreater=0;

\#\#To solve the dual of the Weight\_Corporation

\begin{Shaded}
\begin{Highlighting}[]
\NormalTok{WC\_Dual }\OtherTok{\textless{}{-}} \FunctionTok{read.lp}\NormalTok{(}\StringTok{"Dual\_weigelt.lp"}\NormalTok{)}

\FunctionTok{set.bounds}\NormalTok{(WC\_Dual, }\AttributeTok{lower =} \FunctionTok{c}\NormalTok{(}\DecValTok{0}\NormalTok{,}\DecValTok{0}\NormalTok{,}\DecValTok{0}\NormalTok{,}\DecValTok{0}\NormalTok{,}\DecValTok{0}\NormalTok{,}\DecValTok{0}\NormalTok{,}\DecValTok{0}\NormalTok{,}\DecValTok{0}\NormalTok{,}\DecValTok{0}\NormalTok{,}\SpecialCharTok{{-}}\ConstantTok{Inf}\NormalTok{,}\SpecialCharTok{{-}}\ConstantTok{Inf}\NormalTok{), }\AttributeTok{upper =} \FunctionTok{rep}\NormalTok{(}\ConstantTok{Inf}\NormalTok{,}\DecValTok{11}\NormalTok{))}

\FunctionTok{solve}\NormalTok{(WC\_Dual)}
\end{Highlighting}
\end{Shaded}

\begin{verbatim}
## [1] 0
\end{verbatim}

\begin{Shaded}
\begin{Highlighting}[]
\FunctionTok{get.objective}\NormalTok{(WC\_Dual)}
\end{Highlighting}
\end{Shaded}

\begin{verbatim}
## [1] 696000
\end{verbatim}

\begin{Shaded}
\begin{Highlighting}[]
\FunctionTok{get.constraints}\NormalTok{(WC\_Dual)}
\end{Highlighting}
\end{Shaded}

\begin{verbatim}
## [1] 420 360 324 460 360 300 780 480 300
\end{verbatim}

\begin{Shaded}
\begin{Highlighting}[]
\FunctionTok{get.variables}\NormalTok{(WC\_Dual)}
\end{Highlighting}
\end{Shaded}

\begin{verbatim}
##  [1]  0.00  0.00  0.00 12.00 20.00 60.00  0.00  0.00  0.00 -0.08  0.56
\end{verbatim}

\begin{Shaded}
\begin{Highlighting}[]
\FunctionTok{get.sensitivity.rhs}\NormalTok{(WC\_Dual)}
\end{Highlighting}
\end{Shaded}

\begin{verbatim}
## $duals
##  [1] 516.66667 177.77778   0.00000   0.00000 666.66667 166.66667   0.00000
##  [8]   0.00000 416.66667  55.55556  66.66667  33.33333   0.00000   0.00000
## [15]   0.00000 383.33333 355.55556 166.66667   0.00000   0.00000
## 
## $dualsfrom
##  [1]  3.600000e+02  3.450000e+02 -1.000000e+30 -1.000000e+30  3.450000e+02
##  [6]  2.880000e+02 -1.000000e+30 -1.000000e+30  2.040000e+02 -1.000000e+30
## [11] -2.605325e+13 -1.000000e+30 -1.000000e+30 -1.000000e+30 -1.000000e+30
## [16] -4.000000e+01 -1.500000e+01 -2.400000e+01 -1.000000e+30 -1.000000e+30
## 
## $dualstill
##  [1] 4.60e+02 4.20e+02 1.00e+30 1.00e+30 3.75e+02 3.24e+02 1.00e+30 1.00e+30
##  [9] 1.00e+30 2.52e+02 6.00e+01 4.80e+02 1.00e+30 1.00e+30 1.00e+30 6.00e+01
## [17] 1.50e+01 1.20e+01 1.00e+30 1.00e+30
\end{verbatim}

\begin{Shaded}
\begin{Highlighting}[]
\FunctionTok{get.dual.solution}\NormalTok{(WC\_Dual)}
\end{Highlighting}
\end{Shaded}

\begin{verbatim}
##  [1]   1.00000 516.66667 177.77778   0.00000   0.00000 666.66667 166.66667
##  [8]   0.00000   0.00000 416.66667  55.55556  66.66667  33.33333   0.00000
## [15]   0.00000   0.00000 383.33333 355.55556 166.66667   0.00000   0.00000
\end{verbatim}

\#\#Thus we can see that the Optimal Solution value is same for both
primal and dual problem.

\end{document}
